\chapter*{引言}
\markboth{\textsc{引言}}{}
\addcontentsline{toc}{chapter}{引言}
\setcounter{page}{1}
\pagenumbering{arabic}


\emph{同伦类型论}是一个把若干不同领域以惊人的方式结合在一起的新兴数学分支。它基于近期发现的\emph{同伦论}和\emph{类型论}之间的联系。
同伦论是代数拓扑和同调代数的一个分支,与高阶范畴论有关;而类型论是数理逻辑和理论计算机科学的一个分支。
尽管两者的联系目前是深入研究的焦点,越来越清楚的是,它们只是一个需要花更多时间和更多艰苦工作来完全理解的学科的开端。
这个学科触及看上去相距甚远的主题,例如球体的同伦群,类型检查算法和弱$\infty$-群胚的定义。

同伦类型论也给数学的根本基础问题带来了新的理念。
\index{foundations, univalent}%
一方面,有Voevodsky的微妙而美丽的\emph{一价公理}。 
\index{univalence axiom}%
特别地,一价公理表明可以等同互相同构的结构,这是一个数学家在工作中愉快地使用的原则,尽管它违背“官方”的常规数学基础的教条。
另一方面,我们有\emph{高阶导出类型},它为同伦论中一些基本的空间和构造,如球体,柱体,截面,局部化等,提供直接的逻辑描述。这两点都无法在经典集合论基础下直接实现,当它们在同伦类型论中组合在一起时,则允许实现一种全新的“同伦类型逻辑”。
\index{foundations}%

这提出了一个数学基础的新概念,具有内在的同伦内容,数学对象的“不变”概念,以及方便的机器实现,可以作为数学家工作的实用助理。
这就是\emph{一价基础}项目。本书意图成为一价基础理论的基本内容的第一本系统阐述,以及新型推理风格的例子的一个合集————但不需要读者了解或学习任何形式逻辑,或使用任何计算机证明助理软件。

% This enlarges the page by one line in letter format. Use sparringly.
\OPTwidow

我们强调,同伦类型论是一个年轻的领域,一价基础在很大程度上仍是一个还在进行中的工作。
这本书应被视为只是该领域的一部分的一个快照,拍摄于成书时,而不是对一座完整的大厦的一个完善的阐述。
就像我们稍后将简短讨论的那样,同伦类型论有很多尚未完全被理解的方面————其中甚至有一些方面不会在本书中触及。
最终的理论几乎肯定不会像本书中描述的那样,但\emph{至少}会同样强大有力;因此,我们相信,一价基础最终将成为集合论的一个可行替代方案,作为大多数数学家所做的非形式化数学工作的“隐式基础”。

\subsection*{类型论}

类型论最初是由Bertrand Russell \cite{Russell:1908},\index{Russell, Bertrand}发明的,作为一种消除当时正在研究的数学逻辑基础中的悖论的手段。
在接下来的几十年中,许多人发展了它,特别是Church~\cite{Church:1940tu,Church:1941tc},他将其与他的\textit{$\lambda$-演算}结合起来。
尽管类型论通常未被视为经典数学的基础,更常见的是以集合论为基础,它仍然有很多应用,特别是在计算机科学和程序语言理论~\cite{Pierce-TAPL}中。
\index{programming}%
\index{type theory}%
\index{lambda-calculus@$\lambda$-calculus}%
Per Martin-L\"{o}f \cite{Martin-Lof-1972,Martin-Lof-1973,Martin-Lof-1979,martin-lof:bibliopolis}和其他人开发了Church类型系统的一个“谓词性的”调整,现在通常称为依赖,构造性,直觉,或简单称为\emph{Martin\--L\"of类型论}。这是我们在此考察的系统的基础;它原本意图作为构造性数学形式化的一个严格框架。在接下来,我们将用“类型论”指代该系统或类似系统,虽然类型论作为一个学科要远比这广泛(类型论的历史可见于\cite{somma,kamar})。

在类型论中,与集合论不同,对象使用原始概念\emph{类型}进行分类,和程序语言中使用的数据类型相似。 这些精心构造的类型可用于表示分类对象的详细规范,从而产生关于这些对象的推理原则。 举一个非常简单的例子,我们知道乘积类型$A\times B$的对象形如$\pairr{a,b}$,因此自动地知道如何去构造和分解它们。 类似地,函数类型$A\to B$的对象可由类型$B$的对象在类型$A$的对象中的参数化得到,用类型$A$的对象作为参数取值。 所有对象的这种严格可预测的行为(与集合论更自由的允许非齐次集合的形成原则相反)是类型理论的一个方面,导致其广泛用于验证计算机程序的正确性。  与类型构造相关的明确的推理原则也形成了现代\emph{计算机证明助理}的基础,%
\index{proof!assistant}%
\indexsee{computer proof assistant}{proof assistant}
\index{mathematics!formalized}%
被应用于数学形式化和形式化证明的正确性验证。 下面我们回到类型论的这一方面。

然而,从数学角度理解类型论却总有一个难题:\emph{类型}的基本概念不像\emph{集合}一样,前者在某些方面难以精确定义。 我们认为,从同伦理论的角度,将类型视为空间,而不是奇怪的集合(可能是在不使用经典逻辑的情况下构建的),是向前迈出的重要一步。特别是,它解决了理解类型元素的相等概念与集合元素的相等的概念的差异的问题。

在同伦论中人们关心空间
\index{topological!space}%
及其之间互不同伦的连续映射。
\index{function!continuous!in classical homotopy theory}%
一对连续映射$f : X \to Y$和$g : X\to Y$的\emph{同伦}
\index{homotopy!topological}%
是连续映射$H : X \times [0, 1] \to Y$, 满足
$H(x, 0) = f (x)$以及$H(x, 1) = g(x)$. 同伦$H$可想象为$f$到$g$的连续变形。 空间$X$和$Y$称为\emph{同伦等价的},
\index{homotopy!equivalence!topological}%
$\eqv X Y$, 若存在来回的连续映射,使其复合都同伦于恒等映射,若它们“在同伦意义下”是同构的。 同伦等价的空间有相同的代数不变量(如同调或基本群),它们称为具有相同的\emph{同伦类型}。

\subsection*{同伦类型论}

同伦类型论 (HoTT) 从同伦角度解释类型论。
在同伦类型论中,我们把类型视为“空间”(就像在同伦论中研究的那样)或高阶群胚,逻辑构造(如乘积$A\times B$)则视为在上述空间上的同伦不变构造。
通过这种方式,我们不需要首先发展点集拓扑学(或其组合替代品,如单纯形集合论)就可以直接操纵空间。
为了简要说明该观点,首先考察类型论的基本概念,即\emph{项}$a$具有\emph{类型}$A$,记为:
\[ a:A. \]
这种表达传统上被认为类似于:
\begin{center}
“$a$是集合$A$的一个元素”。
\end{center}
然而,在同伦类型论中,我们认为它是:
\begin{center}
“$a$是空间$A$的一个点”。
\end{center}
\index{continuity of functions in type theory@``continuity'' of functions in type theory}%
类似地,在类型论中的每个函数$f : A\to B$被视为从空间$A$到空间$B$的一个连续映射。

我们需要强调这些“空间”被以纯同伦而非拓扑地处理。
例如,不存在类型的“开子集”或类型元素序列的“收敛”的概念。
我们只有“同伦”概念,如点之间的道路和道路之间的同伦,这在同伦理论的其他模型(如单纯集)中也有意义。
因此,更准确地说,我们将类型视为\emph{$\infty$-群胚}\index{.infinity-groupoid@$\infty$-groupoid};这是同伦理论中“不变对象”的名称,它可以由拓扑空间,
\index{topological!space}%
单纯集或同伦论的任何其他模型表示。
然而,有时使用“空间”和“路径”等拓扑词是方便的,只要我们记住其他拓扑概念不适用。

(在这些对象中使用“同伦类型”
\index{homotopy!type}%
这一术语也很诱人,这表明了对“从同伦角度看的类型(如类型论中的类型)”和“从同伦论角度看的空间”的双重解释。
后者与作为模同伦等价空间的\emph{等价类}的“同伦类型”的经典含义有点不同,尽管它保留了诸如“这两个空间具有相同的同伦类型”之类的术语的原意。)

将类型解释为结构化对象而不是集合的想法由来已久,并以澄清类型论的各种神秘方面而闻名。
例如,将类型解释为层有助于解释类型理论逻辑的直觉性质,而将它们解释为部分等价关系或“域”有助于说明其计算方面。
这也意味着我们可以使用类型论推理来研究结构化对象,从而形成丰富的范畴逻辑领域。
同伦解释也符合相同的模式:它阐明了类型论中的\emph{恒等}(或相等)性质,从而允许我们在同伦论研究中使用类型论推理。

同伦解释的关键思想是,具有同一类型$A$的两个对象$a, b: A$之间的等式$a = b$可以理解为在空间$A$中存在点$a$到点$b$的一条道路$p : a \leadsto b$.
这也意味着两个函数$f, g: A\to B$可被等同,若它们是同伦的,因为同伦只是$B$中的(连续)道路族$p_x: f(x) \leadsto g(x)$, 其中每个$x:A$对应一条道路。
在类型论中,对每个类型$A$, 存在(以前有点神秘的)$A$中两对象的恒等类型$\idtypevar{A}$; 在同伦论中,这只是\emph{道路空间}$A^I$, 表示全体从单位区间发出的连续映射$I\to A$.
\index{unit!interval}%
\index{interval!topological unit}%
\index{path!topological}%
\index{topological!path}%
以这种方式,项$p : \idtype[A]{a}{b}$表示$A$中的一条道路$p : a \leadsto b$. 

同伦类型论的思想出现在2006年Awodey和Warren~\cite{AW}以及Voevodsky~\cite{VV}互相独立的工作中,但也受到Hofmann和Streicher早一些的群胚解释~\cite{hs:gpd-typethy}的启发。
事实上,已知高维范畴论(特别是弱$\infty$-群胚理论)和同伦论有紧密联系,这点由Grothendieck提出,而现在这两方面的数学家正在紧张地研究。
Awodey--Warren和Voevodsky的原始语义模型使用同伦论中众所周知的符号和技术,它们现在也被用在高阶范畴论中,如Quillen模型范畴和Kan\index{Kan complex} 单纯集\index{simplicial!sets}。
\index{Quillen model category}%
\index{model category}%

特别地,Voevodsky基于Kan单纯集构造了一个类型论解释,意识到这种解释还满足一个关键性质,他将其戏称为\emph{一价}性。
这在之前的类型论中没有考虑过(尽管Church的命题可拓性原理它的一个极其特殊的情况,而Hofmann和Streicher考虑过另一个名为“宇宙延展性”的特殊情况)。
以新公理的形式在类型论中加入一价具有深远的影响,其中许多是自然的、简化的和令人信服的。
一价公理还进一步加强了类型论的同伦观点,因为它在单纯模型和其他相关模型中成立,但在类型作为集合的观点下不成立。

\subsection*{一价基础}

长话短说,一价公理的基本思路可以解释如下。
在类型论中,可以有\emph{宇宙}类型$\UU$, 其中的项本身也是类型,例如$A : \UU$等等。
那些是$\UU$的项的类型通常称为\emph{小}类型。
\index{type!small}%
\index{small!type}%
像任何类型一样,$\UU$有一个相等类型$\idtypevar{\UU}$, 表示小类型之间的相等关系$A = B$.
把类型想成空间,$\UU$是空间,上面的点也是空间,为了理解它的相等类型,我们必须问,$\UU$中的空间之间的道路$p : A \leadsto B$是什么?
一价公理是说,(粗略来说)就像上面解释的一样,这些道路对应于同伦等价$\eqv A B$.
更精确一点来说,给定任何(小)类型$A$和$B$, 再加上$A$和$B$之间的原始相等类型$\idtype[\UU]AB$, 就可以定义$A$到$B$的等价类型$\texteqv AB$.
由于任何对象上的相等映射都是等价,因此有典型映射,
\[\idtype[\UU]AB\to\texteqv AB.\]
一价公理指出这个映射自己就是一个等价。
冒着过度简化的风险,我们可以将其简洁地描述如下:

\begin{description}\index{univalence axiom}%
\item[一价公理:]  $\eqvspaced{(A = B)}{(\eqv A B)}$.
\end{description}
%
换句话说,相等等价于等价。
\index{identity}% 
特别地,可以说“等价的类型是相同的”。
然而,这个短语有一些误导性,因为它听起来像是一种把等价的概念\emph{坍缩}以使之与相等吻合的“骨干性”条件,但事实上一价是\emph{扩展}相等概念以与(没有变化的)等价概念吻合。

从同伦观点来看,根据(小)空间的分类空间的直觉,一价性表明具有相同同伦类型的空间之间有宇宙$\UU$中的道路连通。
然而,从逻辑观点看,这是一个激进的新想法:它说彼此同构的事物可以等同!数学家在实践中当然习惯于把同构的结构等同,但他们一般是在知道涉及的对象并不是“真正”相同的前提下,通过“滥用符号”\index{abuse!of notation}或其他非正式的手段实现的。但在这一新基础方案中,这些结构可以正式等同,从逻辑意义上讲,涉及一个结构的每个属性或构造也适用于另一个结构。事实上,这个等同现在已经可以精确作出,可以沿用它以系统地转化性质和构造。此外,这种等同的不同方式本身形成了一种结构,人们可以(也应该!)\ 考虑到这一点。

因此,总而言之,对于宇宙$\UU$的点(即小类型)$A$和$B$, 一价公理把以下三个概念等同:
\begin{itemize}
\item (逻辑的) $A$和$B$的相等$p:A=B$
\item (拓扑的) $\UU$中一条从$A$到$B$的一条道路$p:A \leadsto B$
\item (同伦的) $A$和$B$之间的一个等价$p:\eqv A B$.
\end{itemize}

\subsection*{高阶导出类型}\index{type!higher inductive}%

类型理论的一个经典优点是它处理由归纳定义的结构的简单有效技术。
最简单的非平凡的由归纳定义的结构是自然数,它由零和后继函数归纳生成。
从这句话中,我们可以从算法\index{algorithm}上获取数学归纳法原理,而这表征了自然数。
更一般的归纳定义包括各种类型的列表和良构造的树,每一种都有相应的“归纳原理”。
这包括特定的编程语言中使用的大多数数据结构;因此,类型论在关于后者的形式推理中是有用的。
若从非常普遍的意义考虑,归纳定义也包括像不交并$A+B$这样的例子,它可以被当成由两个单射$A\to A+B$和$B\to A+B$“归纳”生成的。
这种情况下的“归纳原理”是“分情况分析证明”,这给出了不交并的特征。

In homotopy theory, it is natural to consider also ``inductively defined spaces'' which are generated not merely by a collection of \emph{points}, but also by collections of \emph{paths} and higher paths.
Classically, such spaces are called \emph{CW complexes}.
\index{CW complex}%
For instance, the circle $S^1$ is generated by a single point and a single path from that point to itself.
Similarly, the 2-sphere $S^2$ is generated by a single point $b$ and a single two-dimensional path from the constant path at $b$ to itself, while the torus $T^2$ is generated by a single point, two paths $p$ and $q$ from that point to itself, and a two-dimensional path from $p\ct q$ to $q\ct p$.

By using the identification of paths with identities in homotopy type theory, these sort of ``inductively defined spaces'' can be characterized in type theory by ``induction principles'', entirely analogously to classical examples such as the natural numbers and the disjoint union.
The resulting \emph{higher inductive types}
\index{type!higher inductive}%
give a direct ``logical'' way to reason about familiar spaces such as spheres, which (in combination with univalence) can be used to perform familiar arguments from homotopy theory, such as calculating homotopy groups of spheres, in a purely formal way.
The resulting proofs are a marriage of classical homotopy-theoretic ideas with classical type-theoretic ones, yielding new insight into both disciplines.

Moreover, this is only the tip of the iceberg: many abstract constructions from homotopy theory, such as homotopy colimits, suspensions, Postnikov towers, localization, completion, and spectrification, can also be expressed as higher inductive types.
Many of these are classically constructed using Quillen's ``small object argument'', which can be regarded as a finite way of algorithmically describing an infinite CW complex presentation\index{presentation!of a space as a CW complex} of a space, just as ``zero and successor'' is a finite algorithmic\index{algorithm} description of the infinite set of natural numbers.
Spaces produced by the small object argument are infamously complicated and difficult to understand; the type-theoretic approach is potentially much simpler, bypassing the need for any explicit construction by giving direct access to the appropriate ``induction principle''.
Thus, the combination of univalence and higher inductive types suggests the possibility of a revolution, of sorts, in the practice of homotopy theory.


\subsection*{一价基础中的集合}

\index{set|(}%

We have claimed that univalent foundations can eventually serve as a foundation for ``all'' of mathematics, but so far we have discussed 
only homotopy theory.  Of course, there are many specific examples of the use of type theory without the new homotopy type theory features to formalize mathematics,
\index{mathematics!formalized}%
\index{theorem!Feit--Thompson}%
\index{theorem!odd-order}%
\index{Feit--Thompson theorem}%
\index{odd-order theorem}%
such as the recent formalization of the Feit--Thompson odd-order theorem in \Coq~\cite{gonthier}.

But the traditional view is that mathematics is founded on set theory, in the sense that all mathematical objects and constructions can be coded into a theory such as Zermelo--Fraenkel set theory (ZF).
\index{set theory!Zermelo--Fraenkel}%
\indexsee{Zermelo-Fraenkel set theory}{set theory}%
\indexsee{ZF}{set theory}%
\indexsee{ZFC}{set theory}%
However, it is well-established by now that for most mathematics outside of set theory proper, the intricate hierarchical membership structure of sets in ZF is really unnecessary: a more ``structural'' theory, such as Lawvere's\index{Lawvere} Elementary Theory of the Category of Sets~\cite{lawvere:etcs-long}, suffices.
\index{Elementary Theory of the Category of Sets}%

In univalent foundations, the basic objects are ``homotopy types'' rather than sets, but we can \emph{define} a class of types which behave like sets.
Homotopically, these can be thought of as spaces in which every connected component is contractible, i.e.\ those which are homotopy equivalent to a discrete space.
\index{discrete!space}%
It is a theorem  that the category of such ``sets'' satisfies Lawvere's\index{Lawvere} axioms (or related ones, depending on the details of the theory).
Thus, any sort of mathematics that can be represented in an ETCS-like theory (which, experience suggests, is essentially all of mathematics) can equally well be represented in univalent foundations.  

This supports the claim that univalent foundations is at least as good as existing foundations of mathematics.
A mathematician working in univalent foundations can build structures out of sets in a familiar way, with more general homotopy types waiting in the foundational background until there is need of them.
For this reason, most of the applications in this book have been chosen to be areas where univalent foundations has something \emph{new} to contribute that distinguishes it from existing foundational systems.

Unsurprisingly, homotopy theory and category theory are two of these, but perhaps less obvious is that univalent foundations has something new and interesting to offer even in subjects such as set theory and real analysis.
For instance, the univalence axiom allows us to identify isomorphic structures, while higher inductive types allow direct descriptions of objects by their universal properties.
Thus we can generally avoid resorting to arbitrarily chosen representatives or transfinite iterative constructions.
In fact, even the objects of study in ZF set theory can be characterized, inside the sets of univalent foundations, by such an inductive universal property.

\index{set|)}%


\subsection*{Informal type theory}

\index{mathematics!formalized|(defstyle}%
\index{informal type theory|(defstyle}%
\index{type theory!informal|(defstyle}%
\index{type theory!formal|(}%
One difficulty often encountered by the classical mathematician when faced with learning about type theory is that it is usually presented as a fully or partially formalized deductive system.
This style, which is very useful for proof-theoretic investigations, is not particularly convenient for use in applied, informal reasoning.
Nor is it even familiar to most working mathematicians, even those who might be interested in foundations of mathematics.
One objective of the present work is to develop an informal style of doing mathematics in univalent foundations that is at once rigorous and precise, but is also closer to the language and style of presentation of everyday mathematics.

In present-day mathematics, one usually constructs and reasons about mathematical objects in a way that could in principle, one presumes, be formalized in a system of elementary set theory, such as ZFC --- at least given enough ingenuity and patience.
For the most part, one does not even need to be aware of this possibility, since it largely coincides with the condition that a proof be ``fully rigorous'' (in the sense that all mathematicians have come to understand intuitively through education and experience).
But one does need to learn to be careful about a few aspects of ``informal set theory'': the use of collections too large or inchoate to be sets; the axiom of choice and its equivalents; even (for undergraduates) the method of proof by contradiction; and so on.
Adopting a new foundational system such as homotopy type theory as the \emph{implicit formal basis} of informal reasoning will require adjusting some of one's instincts and practices.
The present text is intended to serve as an example of this ``new kind of mathematics'', which is still informal, but could now in principle be formalized in homotopy type theory, rather than ZFC, again given enough ingenuity and patience.

It is worth emphasizing that, in this new system, such formalization can have real practical benefits.
The formal system of type theory is suited to computer systems and has been implemented in existing proof assistants.
\index{proof!assistant}%
A proof assistant is a computer program which guides the user in construction of a fully formal proof, only allowing valid steps of reasoning.
It also provides some degree of automation, can search libraries for existing theorems, and can even extract numerical algorithms\index{algorithm} \index{extraction of algorithms} from the resulting (constructive) proofs.

We believe that this aspect of the univalent foundations program distinguishes it from other approaches to foundations, potentially providing a new practical utility for the working mathematician.
Indeed, proof assistants based on older type theories have already been used to formalize substantial mathematical proofs, such as the four-color theorem\index{theorem!four-color} \index{four-color theorem} and the Feit--Thompson theorem.
Computer implementations of univalent foundations are presently works in progress (like the theory itself).
\index{proof!assistant}%
However, even its currently available implementations (which are mostly small modifications to existing proof assistants such as \Coq and 
\Agda) have already demonstrated their worth, not only in the formalization of known proofs, but in the discovery of new ones.
Indeed, many of the proofs described in this book were actually \emph{first} done in a fully formalized form in a proof assistant, and are only now being ``unformalized'' for the first time --- a reversal of the usual relation between formal and informal mathematics.

One can imagine a not-too-distant future when it will be possible for mathematicians to verify the correctness of their own papers by working within the system of univalent foundations, formalized in a proof assistant, and that doing so will become as natural as typesetting their own papers in \TeX.
%(Whether this proves to be the publishers' dream or their nightmare remains to be seen.) 
In principle, this could be equally true for any other foundational system, but we believe it to be more practically attainable using univalent foundations, as witnessed by the present work and its formal counterpart.

\index{type theory!formal|)}%
\index{informal type theory|)}%
\index{type theory!informal|)}%
\index{mathematics!formalized|)}%

\subsection*{可构造性} 

\index{mathematics!constructive|(}%

One of the most striking differences between classical\index{mathematics!classical} foundations and type theory is the idea of \emph{proof relevance}, according to which mathematical statements, and even their proofs, become first-class mathematical objects.
In type theory, we represent mathematical statements by types, which can be regarded simultaneously as both mathematical constructions and mathematical assertions, a conception also known as \emph{propositions as types}.
\index{proposition!as types}%
Accordingly, we can regard a term $a : A$ as both an element of the type $A$ (or in homotopy type theory, a point of the space $A$), and at the same time, a proof of the proposition $A$.
To take an example, suppose we have sets $A$ and $B$ (discrete spaces),
\index{discrete!space}%
and consider the statement ``$A$ is isomorphic to $B$''.
In type theory, this can be rendered as:
\begin{narrowmultline*}
  \mathsf{Iso}(A,B) \defeq \narrowbreak
  \sm{f : A\to B}{g : B\to A}\Big(\big(\tprd{x:A} g(f(x)) = x\big) \times \big(\tprd{y:B}\, f(g(y)) = y\big)\Big).
\end{narrowmultline*}
%
Reading the type constructors $\Sigma, \Pi, \times$  here  as ``there exists'', ``for all'', and ``and'' respectively yields the usual formulation of ``$A$ and $B$ are isomorphic''; on the other hand, reading them as sums and products yields the \emph{type of all isomorphisms} between $A$ and $B$!  To prove that $A$ and $B$ are isomorphic, one  constructs a proof $p : \mathsf{Iso}(A,B)$, which is therefore the same  as constructing an isomorphism between $A$ and $B$, i.e., exhibiting a pair of functions $f, g$ together with \emph{proofs} that their composites are the respective identity maps.  The latter proofs, in turn, are nothing but homotopies of the appropriate sorts.  In this way, \emph{proving a proposition is the same as constructing an element of some particular type.}
In particular, to prove a statement of the form ``$A$ and $B$'' is just to prove $A$ and to prove $B$, i.e., to give an element of the type $A\times B$.
And to prove that $A$ implies $B$ is just to find an element of $A\to B$, i.e.\ a function from $A$ to $B$ (determining a mapping of proofs of $A$ to proofs of $B$).

The logic of propositions-as-types is flexible and supports many variations, such as using only a subclass of types to represent propositions.
In homotopy type theory, there are natural such subclasses arising from the fact that the system of all types, just like spaces in classical homotopy theory, is ``stratified'' according to the dimensions in which their higher homotopy structure exists or collapses.
In particular, Voevodsky has found a purely type-theoretic definition of \emph{homotopy $n$-types}, corresponding to spaces with no nontrivial homotopy information above dimension $n$.
(The $0$-types are the ``sets'' mentioned previously as satisfying Lawvere's axioms\index{Lawvere}.)
Moreover, with higher inductive types, we can universally ``truncate'' a type into an $n$-type; in classical homotopy theory this would be its $n^{\mathrm{th}}$ Postnikov\index{Postnikov tower} section.\index{n-type@$n$-type}
Particularly important for logic is the case of homotopy $(-1)$-types, which we call \emph{mere propositions}.
Classically, every $(-1)$-type is empty or contractible; we interpret these possibilities as the truth values ``false'' and ``true'' respectively.

Using all types as propositions yields a very ``constructive'' conception of logic; for more on this, see~\cite{kolmogorov,TroelstraI,TroelstraII}.
For instance, every proof that something exists carries with it enough information to actually find such an object; and every proof that ``$A$ or $B$'' holds is either a proof that $A$ holds or a proof that $B$ holds.
Thus, from every proof we can automatically extract an algorithm;\index{algorithm} \index{extraction of algorithms} this can be very useful in applications to computer programming.

On the other hand, however, this logic does diverge from the traditional understanding of existence proofs in mathematics.
In particular, it does not faithfully represent certain important classical principles of reasoning, such as the axiom of choice and the law of excluded middle.
For these we need to use the ``$(-1)$-truncated'' logic, in which only the homotopy $(-1)$-types represent propositions.

\index{axiom!of choice}%
More specifically, consider on one hand the \emph{axiom of choice}: ``if for every $x: A$ there exists a $y:B$ such that $R(x,y)$, there is a function $f : A\to B$ such that for all $x:A$ we have $R(x, f(x))$.''
The pure propositions-as-types notion of ``there exists'' is strong enough to make this statement simply provable --- yet it does not have all the consequences of the usual axiom of choice.
However, in $(-1)$-truncated logic, this statement is not automatically true, but is a strong assumption with the same sorts of consequences as its counterpart in classical\index{mathematics!classical} set theory.

\index{excluded middle}%
\index{univalence axiom}%
On the other hand, consider the \emph{law of excluded middle}: ``for all $A$, either $A$ or not $A$.''
Interpreting this in the pure propositions-as-types logic yields a statement that is inconsistent with the univalence axiom.
For since proving ``$A$'' means exhibiting an element of it, this assumption would give a uniform way of selecting an element from every nonempty type --- a sort of Hilbertian choice operator.
Univalence implies that the element of $A$ selected by such a choice operator must be invariant under all self-equivalences of $A$, since these are identified with self-identities and every operation must respect identity; but clearly some types have automorphisms with no fixed points, e.g.\ we can swap the elements of a two-element type.
\index{automorphism!fixed-point-free}%
However, the ``$(-1)$-truncated law of excluded middle'', though also not automatically true, may consistently be assumed with most of the same consequences as in classical mathematics.

In other words, while the pure propositions-as-types logic is ``constructive'' in the strong algorithmic sense mentioned above, the default $(-1)$-truncated logic is ``constructive'' in a different sense (namely, that of the logic formalized by Heyting under the name ``intuitionistic''); and to the latter we may freely add the axioms of choice and excluded middle to obtain a logic that may be called ``classical''.
Thus, homotopy type theory is compatible with both constructive and classical conceptions of logic, and many more besides.
\index{logic!constructive vs classical}%
Indeed, the homotopical perspective reveals that classical and constructive logic can coexist, as endpoints of a spectrum of different systems, with an infinite number of possibilities in between (the homotopy $n$-types for $-1 < n < \infty$).
We may speak of ``\LEM{n}'' and ``\choice{n}'', with $\choice{\infty}$ being provable and \LEM{\infty} inconsistent with univalence, while $\choice{-1}$ and $\LEM{-1}$ are the versions familiar to classical mathematicians (hence in most cases it is appropriate to assume the subscript $(-1)$ when none is given).  Indeed, one can even have useful systems in which only \emph{certain} types satisfy such further ``classical'' principles, while types in general remain ``constructive''.\index{excluded middle}\index{axiom!of choice}%%

It is worth emphasizing that univalent foundations does not \emph{require} the use of constructive or intuitionistic logic.\index{logic!intuitionistic}\index{logic!constructive} %
Most of classical mathematics which depends on the law of excluded middle and the axiom of choice can be performed in univalent foundations, simply by assuming that these two principles hold (in their proper, $(-1)$-truncated, form).
However, type theory does encourage avoiding these principles when they are unnecessary, for several reasons.

First of all, every mathematician knows that a theorem is more powerful when proven using fewer assumptions, since it applies to more examples.
The situation with \choice{} and \LEM{} is no different:
type theory admits many interesting ``nonstandard'' models, such as in sheaf toposes,\index{topos} where classicality principles such as \choice{} and \LEM{} tend to fail.
Homotopy type theory admits similar models in higher toposes, such as are studied in~\cite{ToenVezzosi02,Rezk05,lurie:higher-topoi}.
Thus, if we avoid using these principles, the theorems we prove will be valid internally to all such models.

Secondly, one of the additional virtues of type theory is its computable character.
In addition to being a foundation for mathematics, type theory is a formal theory of computation, and can be treated as a powerful programming language.
\index{programming}%
From this perspective, the rules of the system cannot be chosen arbitrarily the way set-theoretic axioms can: there must be a harmony between them which allows all proofs to be ``executed'' as programs.
We do not yet fully understand the new principles introduced by homotopy type theory, such as univalence and higher inductive types, from
this point of view, but the basic outlines are emerging; see, for example,~\cite{lh:canonicity}.
It has been known for a long time, however, that principles such as \choice{} and \LEM{} are fundamentally antithetical to computability, since they assert baldly that certain things exist without giving any way to compute them.
Thus, avoiding them is necessary to maintain the character of type theory as a theory of computation.

Fortunately, constructive reasoning is not as hard as it may seem.
In some cases, simply by rephrasing some definitions, a theorem can be made constructive and its proof more elegant.
Moreover, in univalent foundations this seems to happen more often.
For instance:
\begin{enumerate}
\item In set-theoretic foundations, at various points in homotopy theory and category theory one needs the axiom of choice to perform transfinite constructions.
  But with higher inductive types, we can encode these constructions directly and constructively.
  In particular, none of the ``synthetic'' homotopy theory in \cref{cha:homotopy} requires \LEM{} or \choice{}.
\item In set-theoretic foundations, the statement ``every fully faithful and essentially surjective functor is an equivalence of categories'' is equiv\-a\-lent to the axiom of choice.
  But with the univalence axiom, it is just \emph{true}; see \cref{cha:category-theory}.
\item In set theory, various circumlocutions are required to obtain notions of ``cardinal number'' and ``ordinal number'' which canonically represent isomorphism classes of sets and well-ordered sets, respectively --- possibly involving the axiom of choice or the axiom of foundation.
  But with univalence and higher inductive types, we can obtain such representatives directly by truncating the universe; see \cref{cha:set-math}.
\item In set-theoretic foundations, the definition of the real numbers as equivalence classes of Cauchy sequences requires either the law of excluded middle or the axiom of (countable) choice to be well-behaved.
  But with higher inductive types, we can give a version of this definition which is well-behaved and avoids any choice principles; see \cref{cha:real-numbers}.
\end{enumerate}
Of course, these simplifications could as well be taken as evidence that the new methods will not, ultimately, prove to be really constructive.  However, we emphasize again that the reader does not have to care, or worry, about constructivity in order to read this book.  The point is that in all of the above examples, the version of the theory we give has independent advantages, whether or not \LEM{} and \choice{} are assumed to be available.  Constructivity, if attained, will be an added bonus.\index{constructivity}%

Given this discussion of adding new principles such as univalence, higher inductive types, \choice{}, and \LEM{}, one may wonder whether the resulting system remains consistent.
(One of the original virtues of type theory, relative to set theory, was that it can be seen to be consistent by proof-theoretic means).
As with any foundational system, consistency\index{consistency} is a relative question: ``consistent with respect to what?''
The short answer is that all of the constructions and axioms considered in this book have a model in the category of Kan\index{Kan complex} complexes, due to Voevodsky~\cite{klv:ssetmodel} (see~\cite{ls:hits} for higher inductive types).
Thus, they are known to be consistent relative to ZFC (with as many inaccessible cardinals
\index{inaccessible cardinal}\index{consistency}%
as we need nested univalent universes).
Giving a more traditionally type-theoretic account of this consistency is work in progress (see,
e.g.,~\cite{lh:canonicity,coquand2012constructive}).

We summarize the different points of view of the type-theoretic operations in \cref{tab:pov}.

\begin{table}[htb]
  \centering
  \OPTsmalltable
 \begin{tabular}{lllll}
    \toprule
       Types && Logic & Sets & Homotopy\\ \addlinespace[2pt]
    \midrule
       $A$ && proposition & set & space\\ \addlinespace[2pt]
       $a:A$ && proof & element & point \\ \addlinespace[2pt]
       $B(x)$ && predicate & family of sets & fibration \\ \addlinespace[2pt]
       $b(x) : B(x)$ && conditional proof & family of elements & section\\ \addlinespace[2pt]
       $\emptyt, \unit$ && $\bot, \top$ & $\emptyset, \{ \emptyset \}$ & $\emptyset, *$\\ \addlinespace[2pt]
       $A + B$ && $A\vee B$ & disjoint union & coproduct\\ \addlinespace[2pt]
       $A\times B$ && $A\wedge B$ & set of pairs & product space\\ \addlinespace[2pt]
       $A\to B$ && $A\Rightarrow B$ & set of functions & function space\\ \addlinespace[2pt]
       $\sm{x:A}B(x)$ &&  $\exists_{x:A}B(x)$ & disjoint sum & total space\\ \addlinespace[2pt]
       $\prd{x:A}B(x)$ &&  $\forall_{x:A}B(x)$ & product & space of sections\\ \addlinespace[2pt]
       $\mathsf{Id}_{A}$ && equality $=$ & $\setof{\pairr{x,x} | x\in A}$ & path space $A^I$ \\ \addlinespace[2pt]
    \bottomrule
  \end{tabular}
  \caption{Comparing points of view on type-theoretic operations}\label{tab:pov}
\end{table}

\index{mathematics!constructive|)}%

\subsection*{公开问题} 

\index{open!problem|(}%

For those interested in contributing to this new branch of mathematics, it may be encouraging to know that there are many interesting open questions.

\index{univalence axiom!constructivity of}%
Perhaps the most pressing of them is the ``constructivity'' of the Univalence Axiom, posed by Voevodsky in \cite{Universe-poly}.
The basic system of type theory follows the structure of Gentzen's natural deduction. Logical connectives are defined by their introduction rules, and have elimination rules justified by computation rules. Following this pattern, and using Tait's computability method, originally designed to analyse G\"odel's Dialectica interpretation, one can show the property of \emph{normalization} for type theory. This in turn implies important properties such as decidability of type-checking (a crucial property since type-checking corresponds to proof-checking, and one can argue that we should be able to ``recognize a proof when we see one''), and the so-called ``canonicity\index{canonicity} property'' that any closed term of the type of natural numbers reduces to a numeral. This last property, and the uniform structure of introduction/elimination rules, are lost when one extends type theory with an axiom, such as the axiom of function extensionality, or the univalence axiom. Voevodsky has formulated a precise mathematical conjecture connected to this question of canonicity for type theory extended with the axiom of Univalence: given a closed term of the type of natural numbers, is it always possible to find a numeral and a proof that this term is equal to this numeral, where this proof of equality may itself use the univalence axiom? More generally, an important issue is whether it is possible to provide a constructive justification of the univalence axiom.
What about if one adds other homotopically motivated constructions, like higher inductive types?
These questions remain open at the present time, although methods are currently being developed to try to find answers.

Another basic issue is the difficulty of working with types, such as the natural numbers, that are essentially sets (i.e., discrete spaces),
\index{discrete!space}%
containing only trivial paths.
At present, homotopy type theory can really only characterize spaces up to homotopy equivalence, which means that these ``discrete spaces'' may only be \emph{homotopy equivalent} to discrete spaces.
Type-theoretically, this means there are many paths that are equal to reflexivity, but not \emph{judgmentally} equal to it (see \cref{sec:types-vs-sets} for the meaning of ``judgmentally'').
While this homotopy-invariance has advantages, these ``meaningless'' identity terms do introduce needless complications into arguments and constructions, so it would be convenient to have a systematic way of eliminating or collapsing them.
% In some cases, the proliferation of such superfluous identity terms makes it very difficult or impossible to formulate what should be a straightforward concept, such as the definition of a (semi-)simplicial type.

A more specialized, but no less important, problem is the relation between homotopy type theory and the research on \emph{higher toposes}%
\index{.infinity1-topos@$(\infty,1)$-topos}
currently happening at the intersection of higher category theory and homotopy theory.
There is a growing conviction among those familiar with both subjects that they are intimately connected.
For instance, the notion of a univalent universe should coincide with that of an object classifier, while higher inductive types should be an ``elementary'' reflection of local presentability.
More generally, homotopy type theory should be the ``internal language'' of $(\infty,1)$-toposes, just as intuitionistic higher-order logic is the internal language of ordinary 1-toposes.
Despite this general consensus, however, details remain to be worked out --- in particular, questions of coherence and strictness remain to be addressed  --- and doing so will undoubtedly lead to further insights into both concepts.

\index{mathematics!formalized}%
But by far the largest field of work to be done is in the ongoing formalization of everyday mathematics in this new system.
Recent successes in formalizing some facts from basic homotopy theory and category theory have been encouraging; some of these are described in \cref{cha:homotopy,cha:category-theory}.
Obviously, however, much work remains to be done.

\index{open!problem|)}%

The homotopy type theory community maintains a web site and group blog at \url{http://homotopytypetheory.org}, as well as a discussion email list.
Newcomers are always welcome!


\subsection*{如何阅读本书}

This book is divided into two parts.
\cref{part:foundations}, ``Foundations'', develops the fundamental concepts of homotopy type theory.
This is the mathematical foundation on which the development of specific subjects is built, and which is required for the understanding of the univalent foundations approach. To a programmer, this is ``library code''.
Since univalent foundations is a new and different kind of mathematics, its basic notions take some getting used to; thus \cref{part:foundations} is fairly extensive.

\cref{part:mathematics}, ``Mathematics'', consists of four chapters that build on the basic notions of \cref{part:foundations} to exhibit some of the new things we can do with univalent foundations in four different areas of mathematics: homotopy theory (\cref{cha:homotopy}), category theory (\cref{cha:category-theory}), set theory (\cref{cha:set-math}), and real analysis (\cref{cha:real-numbers}).
The chapters in \cref{part:mathematics} are more or less independent of each other, although occasionally one will use a lemma proven in another.

A reader who wants to seriously understand univalent foundations, and be able to work in it, will eventually have to read and understand most of \cref{part:foundations}.
However, a reader who just wants to get a taste of univalent foundations and what it can do may understandably balk at having to work through over 200 pages before getting to the ``meat'' in \cref{part:mathematics}.
Fortunately, not all of \cref{part:foundations} is necessary in order to read the chapters in \cref{part:mathematics}.
Each chapter in \cref{part:mathematics} begins with a brief overview of its subject, what univalent foundations has to contribute to it, and the necessary background from \cref{part:foundations}, so the courageous reader can turn immediately to the appropriate chapter for their favorite subject.
For those who want to understand one or more chapters in \cref{part:mathematics} more deeply than this, but are not ready to read all of \cref{part:foundations}, we provide here a brief summary of \cref{part:foundations}, with remarks about which parts are necessary for which chapters in \cref{part:mathematics}.

\cref{cha:typetheory} is about the basic notions of type theory, prior to any homotopical interpretation.
A reader who is familiar with Martin-L\"of type theory can quickly skim it to pick up the particulars of the theory we are using.
However, readers without experience in type theory will need to read \cref{cha:typetheory}, as there are many subtle differences between type theory and other foundations such as set theory.

\cref{cha:basics} introduces the homotopical viewpoint on type theory, along with the basic notions supporting this view, and describes the homotopical behavior of each component of the type theory from \cref{cha:typetheory}.
It also introduces the \emph{univalence axiom} (\cref{sec:compute-universe}) --- the first of the two basic innovations of homotopy type theory.
Thus, it is quite basic and we encourage everyone to read it, especially \crefrange{sec:equality}{sec:basics-equivalences}.

\cref{cha:logic} describes how we represent logic in homotopy type theory, and its connection to classical logic as well as to constructive and intuitionistic logic.
Here we define the law of excluded middle, the axiom of choice, and the axiom of propositional resizing (although, for the most part, we do not need to assume any of these in the rest of the book), as well as the \emph{propositional truncation} which is essential for representing traditional logic.
This chapter is essential background for \cref{cha:set-math,cha:real-numbers}, less important for \cref{cha:category-theory}, and not so necessary for \cref{cha:homotopy}.

\cref{cha:equivalences,cha:induction} study two special topics in detail: equivalences (and related notions) and generalized inductive definitions.
While these are important subjects in their own rights and provide a deeper understanding of homotopy type theory, for the most part they are not necessary for \cref{part:mathematics}.
Only a few lemmas from \cref{cha:equivalences} are used here and there, while the general discussions in \cref{sec:bool-nat,sec:strictly-positive,sec:generalizations} are helpful for providing the intuition required for \cref{cha:hits}.
The generalized sorts of inductive definition discussed in \cref{sec:generalizations} are also used in a few places in \cref{cha:set-math,cha:real-numbers}.

\cref{cha:hits} introduces the second basic innovation of homotopy type theory --- \emph{higher inductive types} --- with many examples.
Higher inductive types are the primary object of study in \cref{cha:homotopy}, and some particular ones play important roles in \cref{cha:set-math,cha:real-numbers}.
They are not so necessary for \cref{cha:category-theory}, although one example is used in \cref{sec:rezk}.

Finally, \cref{cha:hlevels} discusses homotopy $n$-types and related notions such as $n$-connected types.
These notions are important for \cref{cha:homotopy}, but not so important in the rest of \cref{part:mathematics}, although the case $n=-1$ of some of the lemmas are used in \cref{sec:piw-pretopos}.

This completes \cref{part:foundations}.
As mentioned above, \cref{part:mathematics} consists of four largely unrelated chapters, each describing what univalent foundations has to offer to a particular subject.

Of the chapters in \cref{part:mathematics}, \cref{cha:homotopy} (Homotopy theory) is perhaps the most radical.
Univalent foundations has a very different ``synthetic'' approach to homotopy theory in which homotopy types are the basic objects (namely, the types) rather than being constructed using topological spaces or some other set-theoretic model.
This enables new styles of proof for classical theorems in algebraic topology, of which we present a sampling, from $\pi_1(\Sn^1)=\Z$ to the Freudenthal suspension theorem.

In \cref{cha:category-theory} (Category theory), we develop some basic (1-)category theory, adhering to the principle of the univalence axiom that \emph{equality is isomorphism}.
This has the pleasant effect of ensuring that all definitions and constructions are automatically invariant under equivalence of categories: indeed, equivalent categories are equal just as equivalent types are equal.
(It also has connections to higher category theory and higher topos theory.)

\cref{cha:set-math} (Set theory) studies sets in univalent foundations.
The category of sets has its usual properties, hence provides a foundation for any mathematics that doesn't need homotopical or higher-categorical structures.
We also observe that univalence makes cardinal and ordinal numbers a bit more pleasant, and that higher inductive types yield a cumulative hierarchy satisfying the usual axioms of Zermelo--Fraenkel set theory.

In \cref{cha:real-numbers} (Real numbers), we summarize the construction of Dedekind real numbers, and then observe that higher inductive types allow a definition of Cauchy real numbers that avoids some associated problems in constructive mathematics.
Then we sketch a similar approach to Conway's surreal numbers.

Each chapter in this book ends with a Notes section, which collects historical comments, references to the literature, and attributions of results, to the extent possible.
We have also included Exercises at the end of each chapter, to assist the reader in gaining familiarity with doing mathematics in univalent foundations.

Finally, recall that this book was written as a massively collaborative effort by a large number of people.
We have done our best to achieve consistency in terminology and notation, and to put the mathematics in a linear sequence that flows logically, but it is very likely that some imperfections remain.
We ask the reader's forgiveness for any such infelicities, and welcome suggestions for improvement of the next edition.


% Local Variables:
% TeX-master: "hott-online"
% End:
